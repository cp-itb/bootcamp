\documentclass{article}

\usepackage{geometry}
\usepackage{amsmath}
\usepackage{graphicx, eso-pic}
\usepackage{listings}
\usepackage{hyperref}
\usepackage{multicol}
\usepackage{fancyhdr}
\pagestyle{fancy}
\fancyhf{}
\hypersetup{ colorlinks=true, linkcolor=black, filecolor=magenta, urlcolor=cyan}
\geometry{ a4paper, total={170mm,257mm}, top=40mm, right=20mm, bottom=20mm, left=20mm}
\setlength{\parindent}{0pt}
\setlength{\parskip}{0.3em}
\renewcommand{\headrulewidth}{0pt}
\rfoot{\thepage}
\lfoot{Bootcamp CP ITB 2022}
\lstset{
    basicstyle=\ttfamily\small,
    columns=fixed,
    extendedchars=true,
    breaklines=true,
    tabsize=2,
    prebreak=\raisebox{0ex}[0ex][0ex]{\ensuremath{\hookleftarrow}},
    frame=none,
    showtabs=false,
    showspaces=false,
    showstringspaces=false,
    prebreak={},
    keywordstyle=\color[rgb]{0.627,0.126,0.941},
    commentstyle=\color[rgb]{0.133,0.545,0.133},
    stringstyle=\color[rgb]{01,0,0},
    captionpos=t,
    escapeinside={(\%}{\%)}
}

\begin{document}

\begin{center}
    \section*{Ranking} % ganti judul soal

    \begin{tabular}{ | c c | }
        \hline
        Batas Waktu  & 1s \\    % jangan lupa ganti time limit
        Batas Memori & 256MB \\  % jangan lupa ganti memory limit
        \hline
    \end{tabular}
\end{center}

\subsection*{Deskripsi}
Agus, seorang guru, baru saja selesai menilai jawaban ujian $N$ siswanya. Sekarang ia ingin membuat daftar ranking dari $N$ siswa tersebut. Bantulah Agus!

\subsection*{Format Masukan}
Baris pertama masukan berisi sebuah bilangan bulat $N$ $(1 \leq N \leq 2\times10^5)$ yang menyatakan banyak siswa. \\
Baris kedua masukan berisi $N$ buah bilangan bulat $A_1, A_2, ..., A_N$ $(1 \leq A_i \leq 10^9$ untuk $i \in [1, N])$ yang menyatakan nilai siswa ke-$1$, nilai siswa ke-$2$, dan seterusnya. \textbf{Semua nilai siswa berbeda, tidak ada sepasang nilai siswa yang sama}.

\subsection*{Format Keluaran}
Keluaran terdiri atas satu baris berisi $N$ bilangan bulat. Bilangan ke-$i$ menyatakan ranking siswa ke-$i$ di antara $N$ siswa Agus. Siswa dengan nilai tertinggi memiliki ranking $1$, siswa dengan nilai tertinggi berikutnya memiliki ranking $2$, dan seterusnya hingga siswa dengan nilai terendah memiliki ranking $N$.

\begin{multicols}{2}
\subsection*{Contoh Masukan}
\begin{lstlisting}
5
10 19 2 3 37
\end{lstlisting}
\null
\columnbreak
\subsection*{Contoh Keluaran}
\begin{lstlisting}
3 2 5 4 1
\end{lstlisting}
\vfill
\null
\end{multicols}

\end{document}